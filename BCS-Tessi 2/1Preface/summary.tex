\chapter*{Sommario}
\vspace{0.5cm}
\noindent La presente tesi di laurea triennale descrive la mia esperienza di \textit{stage} svolta presso l'azienda SogeaSoft S.r.l. L'attività di \textit{stage} è durata 304 ore totali svolte tra il 23 settembre e il 17 novembre 2024. 

\vspace{0.5 em}
\noindent L'obiettivo del progetto di \textit{stage} è la migrazione da un'architettura monolitica a un'architettura a microservizi: in particolare, l'azienda richiede un'analisi approfondita del \textit{software} gestionale di SogeaSoft S.r.l. per poi procedere con l'individuazione e l'estrazione dei servizi offerti dal \textit{software} stesso. 

\vspace{0.5 em}
\noindent La presente relazione si suddivide in quattro capitoli:
\begin{itemize}
    \item \textbf{L'azienda}: presentazione dell'azienda dove ho svolto lo \textit{stage} dal punto di vista delle tecnologie utilizzate e dell'organizzazione interna;
    \item \textbf{Progetto di \textit{stage}}: descrizione del progetto di \textit{stage}, del dominio applicativo, degli obiettivi e delle ragioni che hanno motivato questa particolare scelta;
    \item \textbf{Svolgimento dello \textit{stage}}: descrizione delle attività svolte e dei risultati raggiunti;
    \item \textbf{Valutazione retrospettiva}: analisi retrospettiva del lavoro svolto, con particolare riferimento al prodotto finale e ai processi attuati; valutazione della mia evoluzione professionale e personale al termine dello \textit{stage}.
   \end{itemize}

\noindent L'appendice del documento contiene il Glossario dei termini e la lista degli Acronimi. 

\vspace{0.5 em}
\noindent La relazione adotta i seguenti accorgimenti:
\begin{itemize}
    \item i termini in lingua diversa dall'italiano sono segnalati in corsivo, ad eccezione dei nomi propri;
    \item i termini presenti nel Glossario sono caratterizzati da una \textbf{G} a pedice. 
    \item ogni figura è accompagnata da una didascalia che ne descrive il contenuto e, nel caso non sia realizzata da me, la fonte da cui è tratta;
    \item definizioni, regole e affermazioni che non derivano da un ragionamento esplicito nel testo, come ad esempio conclusioni non immediatamente giustificabili, sono accompagnate da un numero ad apice, che indica il riferimento bibliografico da cui provengono. 
\end{itemize}

\clearpage