\chapter{Valutazione retrospettiva}
    \section{Soddisfacimento degli obiettivi di \textit{stage}}
        \subsubsection{Obiettivi aziendali}
        Come anticipato nella Sezione 3.3.2, gli obiettivi redatti inizialmente dal \textit{tutor} aziendale li abbiamo rivalutati in corso d'opera, dando maggiore priorità allo sviluppo \textit{Proof of Concept}. La Tabella \ref{tab:retrospettiva-obiettivi} mostra il grado di completamento degli obiettivi aziendali. 

        \begin{table}[H] \centering \renewcommand{\arraystretch}{1.8} % Increase row height by 50% 
        \begin{tabular}{|>{\bfseries}c|m{13cm}|c|} % Use 'm{}' for vertical centering 
        \hline 
        \multicolumn{3}{|c|}{\textbf{Obiettivi aziendali}} \\ % First row, merged columns 
        \hline 
        \multicolumn{3}{|c|}{\textbf{Obbligatori}} \\ % Second row, merged columns 
        \hline 
        \multirow{2}{*}{\vspace*{\fill}OB1\vspace*{\fill}} & Studio della letteratura esistente sulle architetture monolitiche, sulle architetture a microservizi e sui metodi di migrazione & \checkmark\\ 
        \hline 
        \multirow{2}{*}{\vspace*{\fill}OB2\vspace*{\fill}} & Documentazione relativa ai requisiti & \checkmark \\ 
        \hline 
        \multirow{2}{*}{\vspace*{\fill}OB3\vspace*{\fill}} & Documentazione dei servizi esistenti e delle relazioni tra essi & \xmark\\ 
        \hline 
        \multirow{2}{*}{\vspace*{\fill}OB4\vspace*{\fill}} & Individuazione di un piano indicativo & \checkmark \\ 
        \hline 
        \multicolumn{3}{|c|}{\textbf{Desiderabili}} \\ 
        \hline 
        \multirow{2}{*}{\vspace*{\fill}DE1\vspace*{\fill}} & Documentazione dei rischi & \xmark\\ 
        \hline 
        \multicolumn{3}{|c|}{\textbf{Facoltativi}} \\ 
        \hline 
        \multirow{2}{*}{\vspace*{\fill}FA1\vspace*{\fill}} & Realizzazione del PoC & \checkmark \\ 
        \hline 
        \end{tabular} 
        \caption[Grado di soddisfacimento degli obiettivi aziendali]{Grado di soddisfacimento degli obiettivi aziendali} 
        \label{tab:retrospettiva-obiettivi} 
        \end{table}

        \vspace{0.2 em}
        \noindent Il completamento degli obiettivi indicati in tabella rappresenta un elemento significativo nella progressione del lavoro svolto. Sono riuscita a completare con successo gli obiettivi \textbf{OB1}, \textbf{OB2}, \textbf{OB4} e \textbf{FA1}, consentendo l'estrazione efficace del microservizio.

        \vspace{0.2 em}
        \noindent Ho completamente sviluppato il PoC  e, come evidenziato nella Sezione 2.1, il risultato finale rappresenta un prototipo funzionale, costituendo quindi un'evoluzione significativa rispetto al semplice \textit{Proof of Concept}. Nel corso del lavoro, ho elaborato un'esauriente documentazione sia relativa al PoC che inerente ai requisiti funzionali e tecnici. Inoltre, l'approfondito studio della letteratura scientifica concernente le architetture monolitiche e a microservizi e le metodologie di migrazione l'ho formalizzato in un ampio \textit{report}, ora di proprietà di SogeaSoft S.r.l., che immagino l'azienda utilizzerà come risorsa strategica per proseguire efficacemente il processo di migrazione architetturale.

        \vspace{0.2 em}
        \noindent Per quanto riguarda \textbf{OB3}, sebbene non sia stato soddisfatto completamente, ho sviluppato un'ampia documentazione relativa alla parte dell'ERP esaminata, fornendo linee guida concrete per i progetti futuri. SogeaSoft S.r.l. ha riconosciuto il valore di disporre di un prodotto immediatamente operativo rispetto a concetti puramente teorici, considerando che le soluzioni spesso emergono attraverso processi iterativi di prova ed errore. La realizzazione di risultati tangibili valorizza le ore di lavoro investite e l'impegno dedicato al progetto.

        
        \subsubsection{Obiettivi e aspettative personali}

        In relazione agli obiettivi personali, il livello di soddisfazione che ho raggiunto lo ritengo parzialmente adeguato, come è possibile osservare nella Tabella \ref{tab:retrospettiva-ob-personali}. Il mio approccio iniziale allo \textit{stage} non è stato ottimale, tuttavia nel corso dell'esperienza ho maturato una maggiore consapevolezza che mi ha permesso di valorizzare anche i progressi incrementali. Questo periodo ha rappresentato un momento decisivo nel  mio percorso accademico, costituendo un significativo punto di svolta nella mia carriera universitaria.

        \begin{table}[H]
        \centering
        \renewcommand{\arraystretch}{1.8} % Increase row height by 50%
        \begin{tabular}{|>{\bfseries}c|m{13cm}|c|} % Use 'm{}' for vertical centering
          \hline
          \multicolumn{3}{|c|}{\textbf{Obiettivi personali}} \\ % First row, merged columns
          \hline
          \multirow{2}{*}{\vspace*{\fill}P1\vspace*{\fill}} & Sviluppare maggiore fiducia nelle mie capacità di \textit{problem solving} e ampliare il repertorio di approcci risolutivi & \checkmark \\
          \hline
          \multirow{2}{*}{\vspace*{\fill}P2\vspace*{\fill}} & Approfondire la conoscenza teorica e pratica delle architetture a microservizi, con particolare attenzione ai meccanismi di comunicazione tra servizi & \checkmark \\ 
          \hline
          \multirow{2}{*}{\vspace*{\fill}P3\vspace*{\fill}} & Acquisire la capacità di valutare criticamente le soluzioni tecniche, considerando i compromessi necessari in contesti \textbf{reali} piuttosto che perseguire soluzioni teoricamente perfette & \checkmark \\ 
          \hline
          \multirow{2}{*}{\vspace*{\fill}P4\vspace*{\fill}} & Consolidare la comprensione dei principi di progettazione delle API tra microservizi e delle relative \textit{best practices} implementative & \checkmark \\ 
          \hline
          \multirow{2}{*}{\vspace*{\fill}P5\vspace*{\fill}} & Migliorare nelle competenze di comunicazione professionale, con particolare riferimento alla presentazione tecnica dei risultati durante gli \textit{Sprint review} & \xmark \\ 
          \hline
          \multirow{2}{*}{\vspace*{\fill}P6\vspace*{\fill}} & Migliorare le capacità di comunicazione interpersonale in ambito lavorativo, sia nella richiesta di supporto che nella condivisione dei progressi conseguiti & \xmark \\ 
          \hline
          \multirow{2}{*}{\vspace*{\fill}P7\vspace*{\fill}} & Sviluppare una maggiore consapevolezza del processo di apprendimento professionale, accettando la naturale curva di apprendimento e la progressiva acquisizione di competenze & \checkmark \\ 
          \hline
          \multirow{2}{*}{\vspace*{\fill}P8\vspace*{\fill}} & Ottimizzare la gestione del tempo lavorativo attraverso un approccio più strutturato e consapevole, privilegiando periodi prolungati di \textit{deep work}, ossia concentrazione profonda & \checkmark \\ 
          \hline
        \end{tabular}
        \caption{Grado di soddisfacimento degli obiettivi personali}
        \label{tab:retrospettiva-ob-personali}
        \end{table}


        \noindent La partecipazione a questo \textit{stage} ha notevolmente ampliato la consapevolezza delle mie capacità, che in precedenza non avevo avuto modo di verificare pienamente. L'esperienza mi ha permesso di sviluppare un pensiero autonomo, con il \textit{tutor} aziendale che ha svolto un ruolo importante nell'incoraggiarmi non solo ad esprimere le mie opinioni, ma anche ad accettare la possibilità di sbagliare come parte del processo di apprendimento, almeno durante le attività di progettazione.

        \vspace{0.2 em}
        \noindent Riguardo agli obiettivi di apprendimento e all'utilizzo di nuove tecnologie, l'esperienza è stata molto positiva, permettendomi di approfondire concetti già incontrati in precedenza, come descritto nella Sezione 2.5.1. Ho acquisito una conoscenza più profonda dei principi di progettazione delle API tra microservizi e del protocollo HTTP. Ho anche compreso l'importanza di trovare soluzioni di compromesso efficaci, abbandonando l'ideale di perfezione assoluta.

        \vspace{0.2 em}

        \noindent Diversamente, le mie competenze comunicative non hanno mostrato i miglioramenti sperati. Questo aspetto richiederà ulteriore attenzione, sebbene al momento della stesura di questa tesi io riesca già a vedere alcuni progressi. Durante lo \textit{stage}, ho richiesto supporto più frequentemente, specialmente per l'implementazione di Debezium, mantenendo però una certa autonomia nella ricerca di soluzioni, approccio non sempre ideale soprattutto nei momenti in cui il \textit{tutor} adottava metodologie di lavoro e di comunicazione differenti dalle mie. Le capacità espositive durante le \textit{Sprint Review} sono rimaste sostanzialmente invariate. Nel periodo successivo allo \textit{stage}, ho cercato più occasioni di \textit{public speaking}, con l'intenzione di continuare su questa strada.

        \vspace{0.2 em}
        \noindent Un risultato particolarmente positivo è stato lo sviluppo di sessioni di \textit{deep work}, caratterizzate da concentrazione intensa senza interruzioni. Sono riuscita ad aumentare gradualmente la durata di queste sessioni fino a raggiungere tre periodi giornalieri di novanta minuti ciascuno. Retrospettivamente, questo obiettivo potrebbe aver interferito con la comunicazione tempestiva dei problemi incontrati, evidenziando la necessità di bilanciare l'autonomia operativa con il rispetto delle tempistiche di progetto.
        
    \subsection{Competenze acquisite}
    Come accennavo nella sezione precedente, questa esperienza di \textit{stage} ha completamente cambiato la mia visione verso il mondo professionale ma anche universitario. Oltre ai miei obiettivi personali, ho potuto valutare anche altre competenze che ho acquisito. 

    \subsubsection{Competenze tecniche}
    Sono contenta di constatare di aver acquisito una visione generale del linguaggio C\#. Tuttavia, al di là della specifica tecnologia, ritengo ancora più importante l'aver compreso il funzionamento del \textit{back-end} e, in particolare, la natura del lavoro di integrazione. Inoltre, non avevo mai avuto l'occasione di affrontare processi di \textit{debugging} per progetti di tale complessità, caratterizzati da chiamate distribuite in numerosi punti dell'architettura. Questa esperienza mi ha permesso di consolidare notevolmente le mie competenze tecniche.

    \subsubsection{Competenze organizzative}
    Durante lo \textit{stage} ho sviluppato la capacità di gestire efficacemente il carico di lavoro, distribuendolo strategicamente tra una \textit{Review} e l'altra. Ad esempio, quando necessitavo di supporto non immediatamente disponibile, procedevo con attività indipendenti da quella temporaneamente in attesa. La necessità di organizzare le mie attività per arrivare preparata alle \textit{Review} ha richiesto una pianificazione metodica. 
    
    \vspace{0.2 em}
    \noindent Particolarmente significativo è stato dover affrontare l'essere \textbf{presente} nonostante il disagio e la frustrazione derivanti dalla percezione di eccessiva complessità dei compiti assegnati. È precisamente questa esperienza di resilienza che sottolinea il profondo impatto che questo \textit{stage} ha avuto sul mio percorso professionale, accademico ma soprattutto personale. 

    \subsubsection{Soft Skills}
    Per quanto riguarda le competenze relazionali, devo ammettere che l'esperienza è stata parzialmente limitata. Avrei voluto poter affermare di aver migliorato le mie capacità comunicative o la partecipazione a decisioni tecniche e organizzative, ma le circostanze non lo hanno permesso completamente. In parte per mie limitazioni personali, in parte per l'approccio aziendale, il \textit{tutor} ha adottato uno stile caratterizzato da un controllo costante delle attività, assegnandomi i compiti in modo puntuale e verificandone costantemente l'esecuzione \footnote{\href{https://medium.com/@bkyler2000/micromanagement-7a4c021f3ce1}{https://medium.com/@bkyler2000/micromanagement-7a4c021f3ce1}}. 

    \vspace{0.2 em}
    \noindent Dunque, se inizialmente ho potuto avere una certa autonomia nel decidere come gestire le mie attività, sempre in relazione alle \textit{Sprint Review} che avvenivano con il \textit{Product Owner}, il tipo di comunicazione richiesto dal \textit{tutor} era incompatibile con le \textit{soft skills} che desideravo sviluppare. Il controllo delle attività infatti non mi ha dato occasione di farmi avanti io stessa. Ciononostante, ho potuto trarre beneficio dall'osservazione delle dinamiche interpersonali in un contesto professionale, acquisendo una preziosa visione dell'ambiente lavorativo.
     
    \subsection{Bilancio formativo}
    Le competenze che mi sono state richieste all'inizio dello \textit{stage} non erano distanti rispetto alla mia preparazione accademica. In questo caso l'azienda ha lasciato la formazione iniziale interamente a me, che appunto leggendo dai materiali forniti ho dovuto cercare di capire di cosa si trattasse. Il \textit{tutor} ha fornito il supporto necessario, ma l'assenza di lezioni, anche registrate, che spiegassero almeno in parte il contesto, si è fatta sentire. 

    \vspace{0.2 em}
    \noindent In generale, il percorso universitario, particolarmente nei corsi tecnici e di programmazione, l'ho trovato più che adeguato. L'implementazione di progetti pratici ha contribuito significativamente, poiché mi ha aiutata a sviluppare quella \textit{forma mentis}\footnote{Forma della mente, impostazione del pensiero. \href{https://www.treccani.it/vocabolario/forma-mentis/}{Dizionario online Treccani}} necessaria per l'apprendimento di qualsiasi materia. Anche nel contesto professionale, non esiste una competenza definitiva che garantisca stabilità perpetua: il progresso continuo richiede un costante aggiornamento per evitare l'obsolescenza delle proprie conoscenze. Sotto questo aspetto, ho tratto particolare beneficio dal progetto del corso di Ingegneria del \textit{Software} che, attraverso esperienze esplorative, mi ha permesso di comprendere il funzionamento concreto di un progetto \textit{software}.

    \vspace{0.2 em}
    \noindent Ritengo che i progetti di gruppo previsti dal corso universitario siano sufficientemente calibrati. Ripensando al mio percorso, difficilmente sarei stata pronta ad affrontarne uno durante il primo anno, e anche i progetti individuali del secondo anno hanno rappresentato una sfida considerevole. Il terzo anno, dopo un'introduzione graduale, ha offerto finalmente l'opportunità di confrontarsi con progetti collaborativi.

    \vspace{0.2 em}
    \noindent L'unico elemento che considero carente nel mio percorso formativo è la limitata esposizione a situazioni di comunicazione pubblica. Prima dell'università, non riscontravo difficoltà nell'esporre davanti a un pubblico, grazie alle numerose occasioni offerte sia dal liceo che da altre attività extracurriculari. Riconosco che questa competenza richiede esercizio costante e, in assenza di necessità, tende a non essere sviluppata. Pertanto, credo che maggiori opportunità di esercitare sia l'eloquenza in pubblico sia un atteggiamento proattivo potrebbero rappresentare un valore aggiunto significativo. Una soluzione potrebbe essere la presentazione dei progetti, se non all'intera classe almeno al docente, creando così importanti momenti di confronto e apprendimento.

    \vspace{0.2 em}
    \noindent In alternativa, sarebbe interessante trasformare alcuni esami puramente teorici in prove orali, sempre nell'ottica di sviluppare capacità comunicative. Nonostante l'iniziale disagio che questa modalità potrebbe comportare, i benefici a lungo termine e la soddisfazione personale compenserebbero ampiamente lo sforzo richiesto.

    

    