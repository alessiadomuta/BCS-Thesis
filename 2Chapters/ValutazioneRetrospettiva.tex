\chapter{Valutazione retrospettiva}
    \section{Soddisfacimento degli obiettivi di \textit{stage}}
        \subsubsection{Obiettivi aziendali}
        Come anticipato nella Sezione 3.3.2, gli obiettivi redatti inizialmente dal tutor aziendali sono stati rivalutati in corso d'opera, dando maggiore priorità allo sviluppo \textit{Proof of Concept}. La Tabella \ref{tab:retrospettiva-obiettivi} mostra il grado di completamento degli obiettivi aziendali. 

        \begin{table}[H] \centering \renewcommand{\arraystretch}{1.8} % Increase row height by 50% 
        \begin{tabular}{|>{\bfseries}c|m{13cm}|c|} % Use 'm{}' for vertical centering 
        \hline 
        \multicolumn{3}{|c|}{\textbf{Obiettivi aziendali}} \\ % First row, merged columns 
        \hline 
        \multicolumn{3}{|c|}{\textbf{Obbligatori}} \\ % Second row, merged columns 
        \hline 
        \multirow{2}{*}{\vspace*{\fill}OB1\vspace*{\fill}} & Studio della letteratura esistente sulle architetture monolitiche, sulle architetture a microservizi e sui metodi di migrazione & \checkmark\\ 
        \hline 
        \multirow{2}{*}{\vspace*{\fill}OB2\vspace*{\fill}} & Documentazione relativa ai requsiti & \checkmark \\ 
        \hline 
        \multirow{2}{*}{\vspace*{\fill}OB3\vspace*{\fill}} & Documentazione dei servizi esistenti e delle relazioni tra essi & \xmark\\ 
        \hline 
        \multirow{2}{*}{\vspace*{\fill}OB4\vspace*{\fill}} & Individuazione di un piano indicativo & \checkmark \\ 
        \hline 
        \multicolumn{3}{|c|}{\textbf{Desiderabili}} \\ 
        \hline 
        \multirow{2}{*}{\vspace*{\fill}DE1\vspace*{\fill}} & Documentazione dei rischi & \xmark\\ 
        \hline 
        \multicolumn{3}{|c|}{\textbf{Facoltativi}} \\ 
        \hline 
        \multirow{2}{*}{\vspace*{\fill}FA1\vspace*{\fill}} & Realizzazione del PoC & \checkmark \\ 
        \hline 
        \end{tabular} 
        \caption[Grado di soddisfacimento degli obiettivi aziendali]{Grado di soddisfacimento degli obiettivi aziendali} 
        \label{tab:retrospettiva-obiettivi} 
        \end{table}

        \vspace{0.2 em}
        \noindent Il completamento degli obiettivi indicati in tabella rappresenta un elemento significativo nella progressione del lavoro svolto. Gli obiettivi OB1, OB2, OB4 e FA1 sono stati completati con successo, consentendo l'estrazione efficace del microservizio.

        \vspace{0.2 em}
        \noindent Per quanto riguarda OB3, sebbene non sia stato soddisfatto completamente, è stata sviluppata un'ampia documentazione relativa alla parte dell'ERP esaminata, fornendo linee guida concrete per i progetti futuri. SogeaSoft S.r.l. ha riconosciuto il valore di disporre di un prodotto immediatamente operativo rispetto a concetti puramente teorici, considerando che le soluzioni spesso emergono attraverso processi iterativi di prova ed errore. La realizzazione di risultati tangibili valorizza le ore di lavoro investite e l'impegno profuso nel progetto.

        
        \subsubsection{Obiettivi e aspettative personali}

        In relazione agli obiettivi personali, il livello di soddisfazione raggiunto è parzialmente adeguato, come è possibile osservare nella Tabella \ref{tab:retrospettiva-ob-personali}. L'approccio iniziale allo \textit{stage} non è stato ottimale, tuttavia nel corso dell'esperienza è maturata una maggiore consapevolezza che ha permesso di valorizzare anche i progressi incrementali. Tale periodo ha rappresentato un momento decisivo nel percorso accademico, costituendo un significativo punto di svolta nella mia carriera universitaria.

        \begin{table}[H]
        \centering
        \renewcommand{\arraystretch}{1.8} % Increase row height by 50%
        \begin{tabular}{|>{\bfseries}c|m{13cm}|c|} % Use 'm{}' for vertical centering
          \hline
          \multicolumn{3}{|c|}{\textbf{Obiettivi personali}} \\ % First row, merged columns
          \hline
          \multirow{2}{*}{\vspace*{\fill}P1\vspace*{\fill}} & Sviluppare maggiore fiducia nelle mie capacità di \textit{problem solving} e ampliare il repertorio di approcci risolutivi & \checkmark \\
          \hline
          \multirow{2}{*}{\vspace*{\fill}P2\vspace*{\fill}} & Approfondire la conoscenza teorica e pratica delle architetture a microservizi, con particolare attenzione ai meccanismi di comunicazione tra servizi & \checkmark \\ 
          \hline
          \multirow{2}{*}{\vspace*{\fill}P3\vspace*{\fill}} & Acquisire la capacità di valutare criticamente le soluzioni tecniche, considerando i compromessi necessari in contesti \textbf{reali} piuttosto che perseguire soluzioni teoricamente perfette & \checkmark \\ 
          \hline
          \multirow{2}{*}{\vspace*{\fill}P4\vspace*{\fill}} & Consolidare la comprensione dei principi di progettazione delle API tra microservizi e delle relative \textit{best practices} implementative & \checkmark \\ 
          \hline
          \multirow{2}{*}{\vspace*{\fill}P5\vspace*{\fill}} & Migliorare nelle competenze di comunicazione professionale, con particolare riferimento alla presentazione tecnica dei risultati durante gli \textit{Sprint review} & \xmark \\ 
          \hline
          \multirow{2}{*}{\vspace*{\fill}P6\vspace*{\fill}} & Migliorare le capacità di comunicazione interpersonale in ambito lavorativo, sia nella richiesta di supporto che nella condivisione dei progressi conseguiti & \xmark \\ 
          \hline
          \multirow{2}{*}{\vspace*{\fill}P7\vspace*{\fill}} & Sviluppare una maggiore consapevolezza del processo di apprendimento professionale, accettando la naturale curva di apprendimento e la progressiva acquisizione di competenze & \checkmark \\ 
          \hline
          \multirow{2}{*}{\vspace*{\fill}P8\vspace*{\fill}} & Ottimizzare la gestione del tempo lavorativo attraverso un approccio più strutturato e consapevole, privilegiando periodi prolungati di \textit{deep work}, ossia concentrazione profonda & \checkmark \\ 
          \hline
        \end{tabular}
        \caption{Grado di soddisfacimento degli obiettivi personali}
        \label{tab:retrospettiva-ob-personali}
        \end{table}

        \vspace{0.2 em}
        \noindent La partecipazione a questo \textit{stage} ha notevolmente ampliato la consapevolezza delle mie capacità, che in precedenza non avevo avuto modo di verificare pienamente. L'esperienza mi ha permesso di sviluppare un pensiero autonomo, con il \textit{tutor} aziendale che ha svolto un ruolo importante nell'incoraggiarmi non solo ad esprimere le mie opinioni, ma anche ad accettare la possibilità di sbagliare come parte del processo di apprendimento.

        \vspace{0.2 em}
        \noindent Riguardo agli obiettivi di apprendimento e all'utilizzo di nuove tecnologie, l'esperienza è stata molto positiva, permettendomi di approfondire concetti già incontrati in precedenza, come descritto nella Sezione 2.5.1. Ho acquisito una conoscenza più profonda dei principi di progettazione delle API tra microservizi e del protocollo HTTP. Ho anche compreso l'importanza di trovare soluzioni di compromesso efficaci, abbandonando l'ideale di perfezione assoluta.

        \vspace{0.2 em}

        \noindent Le competenze comunicative non hanno mostrato i miglioramenti sperati. Questo aspetto richiederà ulteriore attenzione, sebbene al momento della stesura di questa tesi siano già visibili alcuni progressi. Durante lo \textit{stage}, ho richiesto supporto più frequentemente, specialmente nell'implementazione di Debezium, mantenendo però una certa autonomia nella ricerca di soluzioni, approccio non sempre ideale in caso di scadenze ravvicinate. Le capacità espositive durante le \textit{Sprint Review} sono rimaste sostanzialmente invariate. Nel periodo successivo allo \textit{stage}, ho cercato più occasioni di \textit{public speaking}, con l'intenzione di continuare su questa strada.

        \vspace{0.2 em}
        \noindent Sono riuscita a superare il senso di inadeguatezza derivante da normali lacune conoscitive, accettando il naturale processo di apprendimento, anche grazie al supporto del \textit{tutor}.

        \vspace{0.2 em}
        \noindent Un risultato particolarmente positivo è stato lo sviluppo di sessioni di \textit{deep work}, caratterizzate da concentrazione intensa senza interruzioni. Sono riuscita ad aumentare gradualmente la durata di queste sessioni fino a raggiungere tre periodi giornalieri di novanta minuti ciascuno. Retrospettivamente, questo obiettivo potrebbe aver interferito con la comunicazione tempestiva dei problemi incontrati, evidenziando la necessità di bilanciare l'autonomia operativa con il rispetto delle tempistiche di progetto.
        
    \subsection{Competenze acquisite}
    Descriverò le competenze e abilità acquisite, valutando la mia evoluzione in modo oggettivo, sia qualitativo che quantitativo. 
    \subsection{Bilancio formativo}
    Descriverò l'eventuale distanza tra la preparazione accademica e le competenze richieste a inizio dello \textit{stage}.