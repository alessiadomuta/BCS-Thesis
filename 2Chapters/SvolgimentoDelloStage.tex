\chapter{Svolgimento dello \textit{stage}}
    \section{Conoscenza del dominio di applicazione}
    Descriverò, seguendo il processo a cui sono stata esposta per lo svolgimento di questo progetto, la prima fase: acquisire familiarità con il dominio di applicazione. Dunque riprenderò la Sezione 2.2.2 descrivendo in breve il "punto di partenza". Introdurrò il singolo modello di Dominio preso in considerazione, meglio descritto nella sezione successiva. 
    \section{Attività svolte}
        \subsection{Analisi dei requisiti}
        Riprendendo la sezione 1.6.1, descriverò nel dettaglio: L'individuazione nel dettaglio del modello di dominio, l'individuazione di \textit{bounded contexts}, i casi d'uso presi in esame, il linguaggio comune utilizzato per una comprensione univoca di tutte le parti coinvolte. 
            \subsubsection{Requisiti}
            Descriverò nel dettaglio i requisiti singoli individuati per i singoli casi d'uso.
        \subsection{Progettazione}
        Riprendendo la Sezione 1.6.2, descriverò nel dettaglio: la modellazione del dominio, la definizione dei microservizi individuati, e i \textit{pattern} individuati per la comunicazione tra essi, nonché per la sincroniccazione dei dati tra monolite e il microservizio preso in esame.
        \subsection{Implementazione}
        Descriverò l'attività di implementazione, riprendendo la sezione relativa nel primo capitolo. 
        Racconterò nel dettaglio i miei \textit{task}, le difficoltà incontrate e le soluzioni ottenute, sia riguardo all'estrazione del microservizio che riguardo la gestione del \textit{database}. 
        Descriverò i \textit{pattern} scelti, come li ho implementati, ed eventuali risultati rilevanti. 
        \subsection{Verifica e Validazione}
        Racconterò il processo di verifica dei risultati raggiunti nella precedente sezione: dai \textit{test} unitari alla validazione con il \textit{team} di sviluppo.
    \section{Risultati raggiunti}
        \subsection{Il Microservizio}
        Descriverò le funzionalità del servizio estratto e l'efficacia dell'aggiornamento dei dati con il monolite.
        Scriverò una visione qualitativa degli obiettivi raggiunti, il loro allineamento al modello di dominio individuato.
        \subsection{Risultati quantitativi}
        Descriverò i risultati quantitativi raggiunti: sia riguardo alla documentazione effettivamente scritta, al codice sviluppato, eventuali \textit{report} e diagrammi, sia riguardo al miglioramento della \textit{performance} del sistema in generale, il \textit{tradeoff} tra miglioramenti e rallentamenti, in base agli obiettivi del progetto (e quindi del dominio).
    \section{Sviluppi futuri}
    Racconterò di quanto il mio progetto di stage sia rilevante per SogeaSoft S.r.l. rispetto al loro obiettivo di migrazione completa del loro sistema verso un'architettura a microservizi.

    
    