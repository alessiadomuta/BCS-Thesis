\chapter{Progetto di \textit{stage}}
    \section{Gestione degli \textit{stage} in SogeaSoft S.r.l.}
    SogeaSoft S.r.l. riconosce il valore strategico degli \textit{stage} curricolari, considerandoli un'opportunità sia per la formazione di potenziali nuovi dipendenti, sia per l'esplorazione di nuove tecnologie e la valutazione critica dei sistemi attualmente in uso. Tali percorsi formativi consentono non solo di trasferire conoscenze e competenze, ma anche di promuovere un'analisi approfondita delle soluzioni tecnologiche adottate dall'azienda, favorendo l'innovazione e l'ottimizzazione dei processi.  

    \vspace{0.2 em}
    Le attività svolte nell'ambito degli \textit{stage} possono includere: 
    \begin{itemize}
        \item l'integrazione di nuove funzionalità nei sistemi esistenti, con l'obiettivo di migliorarne le prestazioni e l'efficienza;
        \item lo studio e lo sviluppo di strumenti autonomi a supporto dei processi di sviluppo o dei prodotti in uso, come ad esempio Swagger, una piattaforma per la documentazione e il testing delle API;  
        \item l'analisi formale di strumenti già utilizzati in azienda, ma impiegati prevalentemente in modo empirico, al fine di standardizzarne e ottimizzarne l’uso;  
        \item la conduzione di attività di monitoraggio sulle \textit{performance} di determinati sistemi, per identificarne eventuali criticità e proporre soluzioni migliorative.  
    \end{itemize}

    \vspace{0.2 em}
    \noindent L'evoluzione tecnologica in SogeaSoft S.r.l. può avvenire attraverso diverse strategie, tra cui l’ampliamento delle funzionalità della \textit{codebase} esistente, l’analisi teorica dello stato dell’arte o la realizzazione di \textit{software} sperimentali, quali \textit{Proof of $Concept_G$} ($PoC_G$) o prodotti già pronti per un utilizzo immediato, come il \textit{Minimum Viable $Product_G$} ($MVP_G$). Nel contesto del mio \textit{stage}, le attività svolte hanno combinato questi approcci, consentendo non solo lo sviluppo di una soluzione concreta, ma anche la produzione di una documentazione tecnica approfondita. Tale metodologia si è rivelata particolarmente efficace per garantire una comprensione strutturata del problema e facilitare eventuali iterazioni successive.  

    \vspace{0.2 em}
    \noindent SogeaSoft S.r.l. attribuisce particolare valore agli \textit{stage}, poiché rappresentano un'opportunità strategica per affrontare una delle sue principali sfide tecnologiche: la migrazione del proprio prodotto da un’architettura $monolitica_G$ a un’architettura a $microservizi_G$, come discusso nella Sezione 1.8. Gli \textit{stage} costituiscono una risorsa vantaggiosa sotto molteplici aspetti: da un lato, permettono di ottimizzare l’investimento in ricerca e sviluppo grazie a costi contenuti; dall’altro, consentono all’azienda di entrare in contatto con prospettive innovative, idee originali e persone non condizionate dai paradigmi consolidati all'interno dell’azienda.  

    \vspace{0.2 em}
    \noindent Un ulteriore fattore determinante nell’impiego di tirocinanti per lo sviluppo di soluzioni innovative è la gestione delle risorse interne. Il personale aziendale è prevalentemente impegnato nel mantenimento e nell’evoluzione dei sistemi attualmente in produzione, rendendo complesso il reindirizzamento delle competenze su progetti sperimentali. L’inserimento di studenti permette di destinare risorse dedicate a iniziative di ricerca e innovazione, garantendo al contempo un processo di trasferimento di conoscenze tra le diverse generazioni di sviluppatori.
    
    \section{Il \textit{software} SAI}
    Descriverò il \textit{software} SAI, oggetto del mio \textit{stage}.
        \subsection{Funzionalità generali di SAI}
        Descriverò le sue funzionalità di base, come viene utilizzato, come funzionano le interazioni tra le parti. (ad alto livello)
        \subsection{Architettura di SAI}
        Descriverò nel dettaglio l'architettura monolitica del sistema, approfondendo (a basso livello) le interazioni tra le parti, la gestione dei dati, per introdurre il capitolo successivo.
    \section{Obiettivi del progetto di \textit{stage}}
        \subsection{Obiettivi aziendali}
        Scriverò degli obiettivi dello \textit{stage}, introducendo il desiderio dell'azienda di migrazione verso un'architettura a microservizi. Racconterò delle motivazioni in modo più approfondito, e riporterò gli obiettivi obbligatori, desiderabili e opzionali decisi all'inizio dell'attività di \textit{stage}.
        \subsection{Vincoli}
        Descriverò i vincoli relativi all'attività di \textit{stage}, saranno temporali e tecnologici. 

    \section{Metodo di Lavoro}
        \subsection{Piainificazione}
        Riporterò il piano di lavoro iniziale e descriverò la suddivisione delle ore, la pianificazione temporale delle attività e il tempo di perseguimento degli obiettivi di avanzamento. 
        \subsection{Modello di sviluppo}
        Rifacendomi alla sezione 1.4, racconterò quanto effettivamente il modello di sviluppo adottato da SogeaSoft S.r.l. sia stato applicato durante il mio \textit{stage} rispetto agli accordi iniziali. 
        \subsection{Strumenti}
        Descriverò, gli strumenti utilizzati a supporto della mia esperienza di \textit{stage}, tra cui gli appunti, le presentazioni, i \textit{report} etc. 
        \subsection{Revisioni di Progresso}
        Riporterò tutte le occasioni di confronto con il \textit{tutor}, le revisioni ufficiali del lavoro svolto ed altri eventuali strumenti messi a disposizione per il monitoraggio del progresso. 
    \section{Motivazioni della scelta}
    Descriverò le motivazioni della mia personale scelta riguardo questo particolare progetto di \textit{stage}, raccontando perché l'ho scelto rispetto ad altri. Descriverò anche gli obiettivi e le aspettative personali, che riprenderò nella sezione 4.1.2
    
        