\chapter{Progetto di \textit{Stage}}
    \section{Gestione degli stage in SogeaSoft S.r.l.}
    Descriverò l'atteggiamento di SogeaSoft S.r.l. nei confronti degli \textit{stage}: l'importanza attribuita, le attività generali dei progetti, come il singolo progetto rappresenti una visione più ampia.
    \section{Il \textit{software} SAI}
    Descriverò il \textit{software} SAI, oggetto del mio \textit{stage}.
        \subsection{Funzionalità generali di SAI}
        Descriverò le sue funzionalità di base, come viene utilizzato, come funzionano le interazioni tra le parti. (ad alto livello)
        \subsection{Architettura di SAI}
        Descriverò nel dettaglio l'architettura monolitica del sistema, approfondendo (a basso livello) le interazioni tra le parti, la gestione dei dati, per introdurre il capitolo successivo.
    \section{Obiettivi del progetto di \textit{stage}}
        \subsection{Obiettivi aziendali}
        Scriverò degli obiettivi dello \textit{stage}, introducendo il desiderio dell'azienda di migrazione verso un'architettura a microservizi. Racconterò delle motivazioni in modo più approfondito, e riporterò gli obiettivi obbligatori, desiderabili e opzionali decisi all'inizio dell'attività di \textit{stage}.
        \subsection{Vincoli}
        Descriverò i vincoli relativi all'attività di \textit{stage}, saranno temporali e tecnologici. 

    \section{Metodo di Lavoro}
        \subsection{Piainificazione}
        Riporterò il piano di lavoro iniziale e descriverò la suddivisione delle ore, la pianificazione temporale delle attività e il tempo di perseguimento degli obiettivi di avanzamento. 
        \subsection{Modello di sviluppo}
        Rifacendomi alla sezione 1.4, racconterò quanto effettivamente il modello di sviluppo adottato da SogeaSoft S.r.l. sia stato applicato durante il mio \textit{stage} rispetto agli accordi iniziali. 
        \subsection{Strumenti}
        Descriverò, gli strumenti utilizzati a supporto della mia esperienza di \textit{stage}, tra cui gli appunti, le presentazioni, i \textit{report} etc. 
        \subsection{Revisioni di Progresso}
        Riporterò tutte le occasioni di confronto con il \textit{tutor}, le revisioni ufficiali del lavoro svolto ed altri eventuali strumenti messi a disposizione per il monitoraggio del progresso. 
    \section{Motivazioni della scelta}
    Descriverò le motivazioni della mia personale scelta riguardo questo particolare progetto di \textit{stage}, raccontando perché l'ho scelto rispetto ad altri. Descriverò anche gli obiettivi e le aspettative personali, che riprenderò nella sezione 4.1.2
    
        