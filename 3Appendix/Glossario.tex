\chapter{Glossario dei termini}
\newenvironment{glossaryitemize}
  {\begin{itemize}[label={}, leftmargin=*]}
  {\end{itemize}}

% Glossary entries grouped by alphabet
\section*{A}
{\color{lightgray}\rule{\textwidth}{0.4pt}} % Thin line separator
\begin{glossaryitemize}
    \item \textbf{Algorithm}: A step-by-step procedure for calculations.
    \item \textbf{Array}: A data structure consisting of a collection of elements.
\end{glossaryitemize}

\section*{B}
{\color{lightgray}\rule{\textwidth}{0.4pt}} % Thin line separator
\begin{glossaryitemize}
    \item \textbf{Binary}: A system of numerical notation with base 2.
    \item \textbf{Boolean}: A data type with two possible values: true or false.
\end{glossaryitemize}

\section*{C}
{\color{lightgray}\rule{\textwidth}{0.4pt}} % Thin line separator
\begin{glossaryitemize}
    \item \textbf{Compiler}: A program that translates code from one language to another.
    \item \textbf{Concatenation}: The operation of joining two strings together.
\end{glossaryitemize}

% Continue for other letters as needed...

% Example of regular itemize with bullets (to confirm it's not affected)
\section*{Regular List}
\begin{itemize}
    \item This is a regular item with a bullet.
    \item Another regular item.
\end{itemize}