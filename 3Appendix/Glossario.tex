\chapter{Glossario dei termini}
\newenvironment{glossaryitemize}
  {\begin{itemize}[label={}, leftmargin=*]}
  {\end{itemize}}

% Glossary entries grouped by alphabet
\section*{A}
{\color{lightgray}\rule{\textwidth}{0.4pt}} % Thin line separator
\begin{glossaryitemize}
    \item \textbf{Advanced Message Queuing Protocol (AMQP)}: protocollo di messaggistica aperto che consente la comunicazione tra applicazioni distribuite.
    \item \textbf{After Sales Service (ASS)}: è l'insieme delle attività di assistenza e supporto fornite al cliente dopo l'acquisto di un prodotto o servizio, con l'obiettivo di garantirne il corretto utilizzo e la soddisfazione.
    \item \textbf{Aggregato}: è un insieme coerente di oggetti del dominio, con un'entità principale che garantisce la consistenza e l'integrità dei dati all'interno dei suoi confini.
    \item \textbf{Anti-Corruption Layer (ACL)}: è un \textit{pattern} architetturale che crea una barriera tra il dominio di un'applicazione e sistemi esterni, prevenendo che logiche o modelli di altri sistemi influenzino la struttura interna del sistema.
    \item \textbf{Application Programming Interface (API)}: insieme di regole e protocolli che consente a diverse applicazioni di comunicare tra loro. Le API definiscono i metodi e le strutture dati necessari per interagire con i servizi e le funzionalità di un sistema.
    \item \textbf{Architettura Esagonale}: è un modello di progettazione \textit{software} che separa il \textit{core} dell'applicazione dalle interfacce esterne, come \textit{database}, servizi o interfacce utente, tramite porte e adattatori.
\end{glossaryitemize}

\section*{B}
{\color{lightgray}\rule{\textwidth}{0.4pt}} % Thin line separator
\begin{glossaryitemize}
    \item \textbf{Bounded Context}: è un'unità autonoma all'interno di un sistema \textit{software}, in cui un modello del dominio ha un significato ben definito e non ambiguo, separato dagli altri contesti per evitare incongruenze.
    \item \textbf{Branch}: è una diramazione indipendente del codice sorgente che consente di sviluppare nuove funzionalità o apportare modifiche senza influenzare la versione principale del progetto.
    \item \textbf{Browser}: \textit{software} per l'acquisizione, la presentazione e la navigazione di risorse sul \textit{Web}.
    \item \textbf{Bug}: difetto o errore nel \textit{software} che causa un comportamento imprevisto o indesiderato, compromettendo il corretto funzionamento del programma.
    \item \textbf{Business Intelligence (BI)}: è un insieme di tecnologie, processi e pratiche che consentono di raccogliere, analizzare e trasformare i dati aziendali in informazioni utili per supportare il processo decisionale.
\end{glossaryitemize}

\section*{C}
{\color{lightgray}\rule{\textwidth}{0.4pt}} % Thin line separator
\begin{glossaryitemize}
    \item \textbf{Change Data Capture (CDC)}: è una tecnica che rileva e traccia le modifiche apportate ai dati in un \textit{database}, permettendo di aggiornare in tempo reale i sistemi che ne fanno uso.
    \item \textbf{Client}: applicazione o dispositivo che richiede servizi o risorse a un \textit{server} in una rete.
    \item \textbf{Command Query Responsibility Segregation (CQRS)}:è un modello architetturale che separa le operazioni di modifica dello stato (comandi) da quelle di lettura dei dati (query).
    \item \textbf{Customer Relationship Management (CRM)}: è un approccio strategico volto a ottimizzare le interazioni e le relazioni con i clienti al fine di migliorare la soddisfazione e la fidelizzazione.
    \item \textbf{Customer Service (CS)}: è il dipartimento o insieme di attività aziendali finalizzate a supportare i clienti prima, durante e dopo l'acquisto di un prodotto o servizio.
\end{glossaryitemize}

\section*{D}
{\color{lightgray}\rule{\textwidth}{0.4pt}} % Thin line separator
\begin{glossaryitemize}
    \item \textbf{Data Transfer Object (DTO)}:  è un oggetto utilizzato per trasportare dati tra processi o livelli di un'applicazione, riducendo il numero di chiamate e semplificando il trasferimento delle informazioni.
    \item \textbf{Database Management System (DBSM)}: sistema che include sia il \textit{software} che le strutture necessarie per gestire e organizzare i dati. Si occupa della creazione, manutenzione e manipolazione dei database, gestendo l'accesso, la sicurezza e l'integrità dei dati.
    \item \textbf{Debito tecnico}: accumulo di problemi o inefficienze nel codice causati da scelte progettuali rapide o soluzioni provvisorie, che richiedono interventi correttivi nel lungo termine.
    \item \textbf{DevOps}: è una pratica che unisce lo sviluppo \textit{software} (Dev) e le operazioni IT (Ops) per migliorare la collaborazione, l'automazione e l'efficienza nel ciclo di vita delle applicazioni, dalla progettazione alla distribuzione e manutenzione.
    \item \textbf{Docker}: piattaforma di containerizzazione che consente di creare, distribuire ed eseguire applicazioni in ambienti isolati chiamati \textit{container}, garantendo coerenza e portabilità tra diversi sistemi.
    \item \textbf{Domain-Driven Design (DDD)}: è un approccio allo sviluppo software che pone al centro la complessità del dominio applicativo, modellandolo attraverso concetti e strutture direttamente ispirate al contesto reale.
    \item \textbf{Domain expert}: è una persona con una conoscenza approfondita di un settore specifico, il cui ruolo è fornire informazioni e guida nello sviluppo di un sistema che rispecchi accuratamente le esigenze del dominio applicativo.
\end{glossaryitemize}

\section*{E}
{\color{lightgray}\rule{\textwidth}{0.4pt}} % Thin line separator
\begin{glossaryitemize}
    \item \textbf{Endpoint (API)}:  è un punto di accesso in un'API che consente di interagire con una risorsa o una funzionalità specifica.
    \item \textbf{Enterprise Resource Planning (ERP)}: è un insieme di strumenti informatici integrati che supportano la gestione e l'automazione dei processi di un'organizzazione. 
    \item \textbf{Entità}: è un oggetto del dominio identificato in modo univoco da un attributo distintivo, indipendentemente dai suoi valori o stato.
    \item \textbf{Event-driven architecture}: è un'architettura in cui le azioni vengono eseguite in risposta a eventi, ossia cambiamenti di stato o segnali che attivano processi specifici, permettendo un'elaborazione asincrona e reattiva.
\end{glossaryitemize}

\section*{F}
{\color{lightgray}\rule{\textwidth}{0.4pt}} % Thin line separator
\begin{glossaryitemize}
    \item \textbf{Feature}: è una caratteristica o funzionalità specifica di un prodotto o sistema, progettata per soddisfare un'esigenza o migliorare l'esperienza dell'utente.
    \item \textbf{Framework}: è un insieme di strumenti, librerie e regole che forniscono una struttura di base per lo sviluppo di applicazioni software, riducendo il lavoro di codifica.
\end{glossaryitemize}

\section*{G}
{\color{lightgray}\rule{\textwidth}{0.4pt}} % Thin line separator
\begin{glossaryitemize}
    \item \textbf{Git}: sistema di controllo versione distribuito che permette di gestire e tracciare le modifiche al codice, facilitando la collaborazione tra sviluppatori e la gestione delle versioni di un progetto.
    item \textbf{Granularità}: in un contesto di sviluppo \textit{software} indica il livello di dettaglio o suddivisione con cui una funzionalità, un modulo o un componente è progettato o implementato.
\end{glossaryitemize}

\section*{H}
{\color{lightgray}\rule{\textwidth}{0.4pt}} % Thin line separator
\begin{glossaryitemize}
    \item \textbf{Handler}: è un componente \textit{software} che gestisce richieste o eventi, traducendoli in operazioni specifiche all'interno di un sistema.
    \item \textbf{HTTP}: è un protocollo di comunicazione utilizzato per trasferire dati su rete, principalmente per il caricamento di pagine \textit{web}, tra \textit{client} (come \textit{browser}) e \textit{server}.
\end{glossaryitemize}

\section*{I}
{\color{lightgray}\rule{\textwidth}{0.4pt}} % Thin line separator
\begin{glossaryitemize}
    \item \textbf{Issue Tracking System (ITS)}: è un \textit{software} utilizzato per gestire, monitorare e risolvere i problemi o le richieste di miglioramento durante lo sviluppo di un progetto, consentendo una gestione efficiente delle attività e delle priorità.
\end{glossaryitemize}

\section*{J}
{\color{lightgray}\rule{\textwidth}{0.4pt}} % Thin line separator
\begin{glossaryitemize}
    \item \textbf{JavaScript Object Notation (JSON)}: è un formato di scambio dati leggero e leggibile, utilizzato per rappresentare strutture dati, spesso impiegato nelle comunicazioni tra \textit{client} e \textit{server}.
\end{glossaryitemize}

\section*{L}
{\color{lightgray}\rule{\textwidth}{0.4pt}} % Thin line separator
\begin{glossaryitemize}
    \item \textbf{Legacy}: nello sviluppo \textit{software}, il termine \textit{\textbf{legacy}} si riferisce a un sistema, codice o tecnologia obsoleta che è ancora in uso, spesso difficile da mantenere o integrare con soluzioni moderne, ma che è fondamentale per il funzionamento dell'organizzazione.
\end{glossaryitemize}

\section*{M}
{\color{lightgray}\rule{\textwidth}{0.4pt}} % Thin line separator
\begin{glossaryitemize}
    \item \textbf{Message Broker}: intermediario che gestisce la ricezione, l'instradamento e la distribuzione di messaggi tra applicazioni o servizi, facilitando la comunicazione tra sistemi.
    \item \textbf{Microservizio}: è un'architettura \textit{software} che suddivide un'applicazione in piccoli servizi autonomi e indipendenti, ognuno dei quali gestisce una specifica funzionalità. Ogni microservizio può essere sviluppato, distribuito e scalato in modo separato, facilitando l'evoluzione e la manutenzione del sistema nel suo complesso.
    \item \textbf{Minimum Viable Product (MVP)}: è una versione base di un prodotto che include solo le funzionalità essenziali, utile per validare l'idea sul mercato con il minimo sforzo di sviluppo.
    \item \textbf{Monolite}: in un contesto di sviluppo \textit{software} è un'applicazione costruita come un unico blocco indivisibile, in cui tutte le funzionalità e componenti sono strettamente interconnessi e distribuiti come un'unica unità.
\end{glossaryitemize}

\section*{N}
{\color{lightgray}\rule{\textwidth}{0.4pt}} % Thin line separator
\begin{glossaryitemize}
    \item \textbf{Namespace}: è uno spazio di denominazione che raggruppa identificatori univoci, come variabili o funzioni, per evitare conflitti di nome all'interno di un programma.
\end{glossaryitemize}

\section*{P}
{\color{lightgray}\rule{\textwidth}{0.4pt}} % Thin line separator
\begin{glossaryitemize}
    \item \textbf{Pattern}: Un \textit{pattern} è una soluzione ripetibile e collaudata a un problema ricorrente, che può essere applicata in vari contesti per risolvere specifiche difficoltà nel processo di progettazione o sviluppo.  I \textit{pattern} architetturali si riferiscono a soluzioni generali per la struttura di un sistema \textit{software}, come il \textit{pattern} a microservizi, che definisce come organizzare i componenti di un sistema. I \textit{pattern} di \textit{design} riguardano soluzioni a problemi di progettazione a livello di componenti. I \textit{pattern} di comportamento si concentrano su come gli oggetti interagiscono tra loro.
    \item \textbf{Payload}: è la parte di un messaggio o di una richiesta che contiene i dati effettivi da elaborare, escludendo intestazioni o informazioni di controllo.
    \item \textbf{Pipeline}: in Azure è un processo automatizzato che consente di compilare, testare e distribuire il codice attraverso diverse fasi, facilitando l'integrazione continua e la consegna continua per le applicazioni.
    \item \textbf{Plan-Do-Check-Act (PDCA)}: metodo iterativo di gestione del lavoro utilizzato per il miglioramento continuo di processi o prodotti. Anche noto come ciclo di Deming.
    \item \textbf{Proof of Concept (PoC)}: è una dimostrazione pratica volta a verificare la fattibilità o l'efficacia di un'idea o soluzione tecnica. Viene utilizzata per validare un concetto prima di investire risorse nello sviluppo completo.
    \item \textbf{Prototipo}: è una versione preliminare di un prodotto o sistema, realizzata per testare e valutare funzionalità, \textit{design} o prestazioni prima della produzione definitiva.
    \item \textbf{Pull Request (PR)}: è una proposta di modifica al codice in un sistema di controllo versione, che consente di revisionare e integrare le modifiche da un ramo a un altro, tipicamente dalla versione di sviluppo alla versione principale.
\end{glossaryitemize}

\section*{Q}
{\color{lightgray}\rule{\textwidth}{0.4pt}} % Thin line separator
\begin{glossaryitemize}
    \item \textbf{Query}: è una richiesta formulata a un \textit{database} per ottenere, inserire, modificare o eliminare dati in base a criteri specifici.
\end{glossaryitemize}

\section*{R}
{\color{lightgray}\rule{\textwidth}{0.4pt}} % Thin line separator
\begin{glossaryitemize}
    \item \textbf{Repository}: è un archivio centralizzato in cui vengono conservati, gestiti e versionati file e codici sorgente di un progetto, spesso utilizzando un sistema di controllo delle versioni.
    \item \textbf{Representational State Transfer (REST)}: modello di architettura \textit{software} creato per guidare la progettazione e lo sviluppo dell'architettura di applicazioni di rete. Valorizza l'uso di interfacce standard (API) e il \textit{deployment} distribuito, indipendente e scalabile.
    \item \textbf{Return On Investment (ROI)}: è un indicatore finanziario che misura la redditività di un investimento, calcolato come il rapporto tra il guadagno ottenuto e il costo sostenuto.
\end{glossaryitemize}

\section*{S}
{\color{lightgray}\rule{\textwidth}{0.4pt}} % Thin line separator
\begin{glossaryitemize}
    \item \textbf{Single-Page Application (SPA)}: applicazione \textit{web} che risponde alle richieste dell'utente sovrascrivendo il contenuto della pagina corrente, anziché seguire il comportamento standard di caricare una nuova pagina.
    \item \textbf{Stakeholder}: tutte le persone, gruppi o organizzazioni che hanno un interesse diretto o indiretto nei risultati di un progetto o di un'azienda. Possono includere clienti, fornitori, dipendenti, azionisti, enti regolatori e la comunità in generale.
    \item \textbf{Structured Query Language (SQL)}: linguaggio standard utilizzato per gestire e manipolare dati in un \textit{database} relazionale.
    \item \textbf{Supplier Relationship Management (SRM)}: è un approccio strategico e gestionale volto a ottimizzare e gestire le relazioni con i fornitori di un'organizzazione.
\end{glossaryitemize}

\section*{U}
{\color{lightgray}\rule{\textwidth}{0.4pt}} % Thin line separator
\begin{glossaryitemize}
    \item \textbf{Ubiquiotous Language}: è un linguaggio comune, condiviso tra sviluppatori ed esperti del dominio, che garantisce coerenza nella comunicazione e nella modellazione del sistema \textit{software}.
    \item \textbf{Uniform Resource Locator (URL)}: è un indirizzo che specifica la posizione di una risorsa su una rete, consentendo di accedervi tramite un protocollo come HTTP.
    \item \textbf{User story}: descrizione breve e informale di una funzionalità dal punto di vista dell'utente finale, utilizzata nello sviluppo \textit{Agile} per definire i requisiti in modo chiaro e orientato al valore.
\end{glossaryitemize}

\section*{V}
{\color{lightgray}\rule{\textwidth}{0.4pt}} % Thin line separator
\begin{glossaryitemize}
    \item \textbf{Value Object}: è un oggetto del dominio che rappresenta un concetto privo di identità univoca, definito solo dai suoi attributi e utilizzato per esprimere valori o proprietà immutabili.
    \item \textbf{Version Control System (VCS)}: è uno strumento che permette di tenere traccia delle modifiche apportate a un progetto nel tempo, consentendo di gestire diverse versioni del codice.
\end{glossaryitemize}



% Continue for other letters as needed...
%\section*{z}
%{\color{lightgray}\rule{\textwidth}{0.4pt}} % Thin line separator
%\begin{glossaryitemize}
%   \item z
%\end{glossaryitemize}